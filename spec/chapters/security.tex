\chapter{Security}
\label{security}

\section{Introduction}

Guarding against malicious attacks is a great concern for web application developers. In
particular, MVC applications that accept input from a browser are often targetted by 
attackers. Two of the most common forms of attacks are cross-site request forgery (CSRF)
and cross-site scripting (XSS). This chapter explores techniques to prevent these type
of attacks with the aid of the MVC API.

\section{Cross-site Request Forgery}

Cross-site Request Forgery (CSRF) is a type of attack in which a user, who has a trust
relationship with a certain site, is mislead into executing some commands that exploit the
existence of such a trust relationship. The canonical example for this attack is that of
a user unintentionally carrying out a bank transfer while visiting another site. 

The attack is based on the inclusion of a link or script in a page that accesses a site 
to which the user is known or assumed to have been authenticated (trusted). Trust
relationships are often stored in the form of cookies that may be active while the user
is visiting other sites. For example, such a malicious site could include the following
HTML snippet:

\begin{verbatim}
<img src="http://yourbank.com/transfer?from=yours&to=mine&sum=1000000">
\end{verbatim}

which will result in the browser executing a bank transfer in an attempt to load an image.

In practice, most sites require the use of form posts to submit requests such as bank
transfers. The common way to prevent CSRF attacks is by embedding additional, 
difficult-to-guess data fields in requests that contain sensible commands. This 
additional data, known as a token, is obtained from the trusted site but 
unlike cookies it is never stored in the browser.

MVC implementations provide CSRF protection using the {\tt Csrf} object
and the {\tt @CsrfValid} annotation. The {\tt Csrf} object is available to applications
via the injectable {\tt MvcContext} type or in EL as {\tt mvc.csrf}. For more information
about {\tt MvcContext}, please refer to Section~\ref{mvc_context}. 

Applications may use the {\tt Csrf} object to inject a hidden field in a form that can be validated 
upon submission. Consider the following JSP,

\begin{listing}{1}
<html>
<head>
    <title>CSRF Protected Form</title>
</head>
<body>
    <form action="csrf" method="post" accept-charset="utf-8">
        <input type="submit" value="Click here"/>
        <input type="hidden" name="${mvc.csrf.name}" 
                             value="${mvc.csrf.token}"/>
    </form>
</html>
\end{listing}

The hidden field will be submitted with the form, giving the MVC implementation
the opportunity to verify the token and ensure the validity of the post request. 

Another way to convey this information to and form the client is via an HTTP
header. MVC implementations are REQUIRED to support CSRF tokens both
as form fields (with the help of the application developer as shown above) and
as HTTP headers.

The application-level property {\tt javax.mvc.security.CsrfProtection} enables
 CSRF protection when set to one of the possible values defined in
{\tt javax.mvc.security.Csrf.CsrfOptions}. The default value of this property is
{\tt CsrfOptions.OFF}. Setting it to any other value will automatically inject a
CSRF token as an HTTP header; the actual name of this header is 
implementation dependent.

Automatic validation is enabled by setting this
property to {\tt CsrfOptions.IMPLICIT}, in which case all post requests
must include either an HTTP header or a hidden field with the correct token.
Finally, if the property is set to {\tt CsrfOptions.EXPLICIT} then application
developers must annotate controllers using {\tt @CsrfValid} to manually
enable validation \assertref{csrf-options} as shown in the following example.

\begin{listing}{1}
@Path("csrf")
@Controller
public class CsrfController {

    @GET
    public String getForm() {
        return "csrf.jsp";     // Injects CSRF token
    }

    @POST
    @CsrfValid		           // Required for CsrfOptions.EXPLICIT
    public void postForm(@FormParam("greeting") String greeting) {
        // Process form
    }
}
 \end{listing}

MVC implementations are required to support CSRF validation of
tokens for controllers annotated with {\tt @POST} and consuming the
media type {\tt x-www-form-urlencoded} \assertref{csrf-support}; other media 
types and scenarios may also be supported but are OPTIONAL.

\section{Cross-site Scripting}

Cross-site scripting (XSS) is a type of attack in which snippets of scripting 
code are injected and later executed when returned back from a server. The
typical scenario is that of a website with a search field that does not validate
its input, and returns an error message that includes the value that was
submitted. If the value includes a snippet of the form {\tt <script>...</script>}
then it will be executed by the browser when the page containing the 
error is rendered.

There are lots of different variations of this the XSS attack, but most can be
prevented by ensuring that the data submitted by clients is properly {\em sanitized}
before it is manipulated, stored in a database, returned to the client, etc.
Data escaping/encoding is the recommended way to deal with untrusted
data and prevent XSS attacks. 

MVC applications can gain access to encoders through the {\tt MvcContext}
object; the methods defined by {\tt javax.mvc.security.Encoders} can be used
by applications to contextually encode data in an attempt to prevent 
XSS attacks. The reader is referred to the Javadoc for this type 
for further information.
