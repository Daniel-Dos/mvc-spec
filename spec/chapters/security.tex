\chapter{Security}

\section{Introduction}

{\em (TBD)}

\section{Cross-site Request Forgery}

Cross-site Request Forgery (CSRF) is a type of attack in which a user, who has a trust
relationship with a certain site, is mislead to execute some commands that exploit the
existence of such a trust relationship. The canonical example for this attack is that of
a user unintentionally carrying out a bank transfer while visiting another site. 

The attack is based on the inclusion of a link or script in a page that accesses a site 
to which the user is known or assumed to have been authenticated (trusted). Trust
relationship are often stored in the form of cookies that may be active while the user
is visiting other sites. For example, such a malicious site could include the following
HTML snippet:

\begin{verbatim}
<img src="http://yourbank.com/transfer?from=yours&to=mine&sum=1000000">
\end{verbatim}

that will mislead a browser to execute a bank transfer in an attempt to load an image.

In practice, most sites require the use of form posts to submit requests such as bank
transfers. The common way to prevent CSRF attacks is by embedding additional, 
difficult-to-guess data fields in requests that contain sensible commands. This 
additional data, or token, is obtained from the trusted site but unlike cookies it is
never stored in the browser.

MVC implementations include support CSRF protection via the use of tokens.
These tokens need to be generated and also validated to ensure applications
are protected from CSRF attacks.

{\em (TBD)}

\section{Cross-site Scripting}

{\em (TBD)}

