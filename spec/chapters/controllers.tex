\chapter{Models, Views and Controllers}
\label{mvc}

This chapter introduces the three components that comprise the \mvc\ architectural 
pattern: models, views and controllers. 

\section{Controllers}
\label{controllers}

An {\em MVC controller} is a \jaxrs\ \cite{jaxrs} resource method decorated by 
an \Controller\ annotation \assertref{controller}. 
If this annotation is applied to a class, then all methods in it are regarded as controllers
\assertref{all-controllers}. 
Using the \Controller\ annotation on a subset of methods defines a hybrid class in which 
certain methods are controllers and others are traditional \jaxrs\ resource methods.

A simple hello-world controller can be defined as follows:

\begin{listing}{1}
@Path("hello")
public class HelloController {

    @GET
    @Controller
    public String hello() {
        return "hello.jsp";
    }
}
\end{listing}

In this example, \code{hello} is a controller method that returns a path to a JavaServer Page (JSP).
The semantics of controller methods differ slightly from \jaxrs\ resource methods; in particular, a
return type of \code{String} is interpreted as a view path rather than text content. Moreover,
the default media type for a response is assumed to be \code{text/html}, but otherwise can
be declared using \Produces\ just like in \jaxrs.

The return type of a controller method is restricted to be one of four possible types
\assertref{controller-types}:
\begin{description}
\item[void] A controller method that returns void is REQUIRED to be decorated by \View\
\assertref{void-controllers}. 
\item[String] The string returned is interpreted as a path to a view. 
\item[Viewable] A \Viewable\ is a class that encapsulates information about a view and
how it is processed.
\item[Response] A \jaxrs\ \Response\ whose entity's type is one of the three above.
\end{description}

The following class defines equivalent controller methods:

\begin{listing}{1}
@Controller
@Path("hello")
public class HelloController {

    @GET
    @View("hello.jsp")
    public void helloVoid() {
    }
    
    @GET
    public String helloString() {
        return "hello.jsp";
    }
    
    @GET
    public Viewable helloViewable() {
        return new Viewable("hello.jsp");
    }
    
    @GET
    public Response helloResponse() {
        return Response.status(Response.Status.OK)
                       .entity("hello.jsp").build();
    }    
}
\end{listing}

Note that, even though controller methods return types are restricted as 
explained above, \mvc\ does not impose any restrictions on parameter
types available to controller methods: i.e., all parameter types injectable
in \jaxrs\ resources are also available in \mvc\ controllers. Likewise, injection
of fields and properties is unrestricted and fully compatible with \jaxrs\ ---modulo
the restrictions explained in Section \ref{controller_instances}.

\subsection{Controller Instances}
\label{controller_instances}

Unlike in \jaxrs\ where resource classes can be native (created and managed by \jaxrs), 
CDI beans, managed beans or EJBs, \mvc\ classes are REQUIRED to be CDI-managed beans 
only \assertref{cdi-beans}. It follows that a hybrid class that contains a mix 
of \jaxrs\ resource methods and \mvc\ controllers must also be CDI managed. 

Like in \jaxrs, the default resource class instance lifecycle is {\em per-request}
\assertref{per-request}.
That is, an instance of a controller class MUST be instantiated and initialized 
on every request. Implementations MAY support other lifecycles via CDI; the same caveats 
that apply to \jaxrs\ classes in other lifecycles applied to \mvc\ classes. In particular, 
proxies may be necessary when, for example, a per-request instance is as a member of a 
per-application instance. See \cite{jaxrs} for more information on lifecycles and 
their caveats.

\subsection{Viewable}
\label{viewable}

The \Viewable\ class encapsulates information about a view as well as, optionally, 
information about how it  should be processed. More precisely, a \Viewable\ instance 
may include references to \Models\ and \ViewEngine\ objects ---for more information 
see Section \ref{models} and Chapter \ref{view_engines}, respectively. 
\Viewable\ defines traditional constructors for all these objects and it is, therefore, 
not a CDI-managed bean.

The reader is referred to the Javadoc of the \Viewable\ class for more information on its semantics.

\subsection{Response}
\label{response}

Returning a \Response\ object gives applications full access to all the parts in a response, 
including the headers. For example, an instance of  \Response\ can modify the HTTP status
code upon encountering an error condition; \jaxrs\ provides a fluent API to build responses
as shown next.

\begin{listing}{1}
@GET
@Controller
public Response getById(@PathParam("id") String id) {
    if (id.length() == 0) {
   	    return Response.status(Response.Status.BAD_REQUEST)
   	                   .entity("error.jsp").build();
    } 
    ...
}
\end{listing}

Direct access to \Response\ enables applications to override content types, set character
encodings, set cache control policies, trigger an HTTP redirect, etc. For more information, 
the reader is referred to the Javadoc for the \Response\ class.

\section{Models}
\label{models}

\mvc\ controllers are responsible for combining data models and views (templates) to 
produce web application pages. This specification supports two kinds of models: the
first is based on CDI \Named\ beans, and the second on the \Models\ interface
which defines a map between names and objects. Support for the \Models\ 
interface is mandatory for all view engines; support for CDI \Named\ beans is
OPTIONAL but highly RECOMMENDED. Application developers are encouraged to use CDI-based
models whenever supported, and thus take advantage of the existing CDI and EL integration
on the platform. 

Let us now revisit our hello-world example, this time also showing how to update
a model. Since we intent to show the two ways in which models can be used, we define the
model as a CDI \Named\ bean in request scope even though this is only necessary
for the CDI case:

\begin{listing}{1}
@Named("greeting")
@RequestScoped
public class Greeting {

	private String message;
	
	public String getMessage() { return message; }
	public void setMessage(String message) { this.message = message; }
	...
}
\end{listing}

Given that the view engine for JSPs supports \Named\ beans, all the controller
needs to do is fill out the model and return the view. Access to the model
is straightforward using CDI injection:

\begin{listing}{1}
@Path("hello")
public class HelloController {

	@Inject
	private Greeting greeting;

    @GET
    @Controller
    public String hello() {
        greeting.setMessage("Hello there!");
        return "hello.jsp";
    }
}
\end{listing}

If the view engine that processes the view returned by the controller is not CDI 
enabled, then controllers can use the \Models\ map instead:

\begin{listing}{1}
@Path("hello")
public class HelloController {

	@Inject
	private Models models;

    @GET
    @Controller
    public String hello() {
        models.put("greeting", new Greeting("Hello there!");
        return "hello.jsp";
    }
}
\end{listing}

In this example, the model is given the same name as that in the \code{@Named}
annotation above, but using the injectable \Models\ map instead.

As stated above, the use of typed CDI \Named\ beans is recommended over the \Models\ map,
but support for the latter may be necessary to integrate view engines that are not
CDI aware. For more information about view engines see Chapter~\ref{view_engines}.

\section{Views}
\label{views}

A {\em view}, sometimes also referred to as a template, defines the structure of the output
page and can refer to one or more models. It is the responsibility of a {\em view engine}
to process (render) a view by extracting the information in the models and producing the
output page. 

Here is the JSP page for the hello-world example:

\begin{listing}{1}
<%@ page contentType="text/html;charset=UTF-8" language="java" %>
<html>
<head>
    <title>Hello</title>
</head>
<body>
<h1>${greeting.message}</h1>
</body>
</html>
\end{listing}

In a JSP, model properties are accessible via EL \cite{el}. In the example above,
the property \code{message} is read from the \code{greeting} model whose name
was either specified in a \Named\ annotation or used as a key in the \Models\ map, 
depending on which controller from Section \ref{models} triggered this view's 
processing.

