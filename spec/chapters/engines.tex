\chapter{View Engines}
\label{view_engines}

This chapter introduces the notion of a view engine as the mechanism by which views
are processed in \mvc. The set of available view engines is extensible via CDI,
enabling applications as well as other frameworks to provide support for additional
view languages.

\section{Introduction}
\label{view_engines_introduction}

A {\em view engine} is responsible for processing views. In this context, processing entails
(i) locating and loading a view (ii) preparing any required models and (iii) rendering the 
view and writing the result back to the client. 

Implementations MUST provide built-in support for JSPs and Facelets view engines. Additional
engines may be supported via an extension mechanism based on CDI. Namely, any CDI bean
that implements the \code{javax.mvc.engine.ViewEngine} interface MUST be considered as
a possible target for processing by calling its \code{support} method, discarding the
engine if this method returns \code{false}.

This is the interface that must be implemented by all \mvc\ view engines:

\begin{listing}{1}
public interface ViewEngine {

    boolean supports(String view);

    void processView(ViewEngineContext context) throws ViewEngineException;
}
\end{listing}

\section{Selection Algorithm}
\label{selection_algorithm}

As explained in Section \ref{viewable}, a \Viewable\ is an encapsulation for information
that relates to a view. Every possible return type from a controller method is either
a \Viewable\ or can be turned into one by calling a constructor. Thus,
the following algorithm assumes only \Viewable\ as input.

Implementations should perform the following steps while trying to find
a suitable view engine for a \Viewable.

\begin{enumerate}
\item If calling \code{getViewEngine} on the \Viewable\ returns a non-null value, return
that view engine.
\item Otherwise, lookup all instances of \code{javax.mvc.engine.ViewEngine} available via CDI.~\footnote{The \code{@Any} annotation in CDI can be used for this purpose.}
\item Call \code{supports} on every view engine found in the previous step, discarding
those that return \code{false}.
\item If the resulting set is empty, return \code{null}.
\item Otherwise, sort the resulting set in descending order of priority using the integer value 
from the \Priority\ annotation decorating the view engine class or the default value \code{Priorities.DEFAULT} if the annotation is not present.
\item Return the first element in the resulting sorted set, that is, the view engine with
the highest priority that supports the given \Viewable.
\end{enumerate}

If a view engine that can process a \Viewable\ is not found, as a fall-back attempt to 
process the view by other means, implementations are 
REQUIRED to forward the request-response pair back to the Servlet container using a 
\code{RequestDispatcher}.

The \code{processView} method has all the information necessary for processing
in the \code{ViewEngineContext},
including the view, a reference to \Models, as well as the HTTP request and response
 from the underlying the Servlet container. Implementations MUST catch
exceptions thrown during the execution of \code{processView} and re-throw them as 
\code{ViewEngineException}'s. 

Prior to the view render phase, all entries available in \Models\ MUST be bound in
such a way that they become available to the view being processed. The exact mechanism
for this depends on the actual view engine implementation. In the case of the built-in
view engines for JSPs and Facelets, entries in \Models\ must be bound by calling
\code{HttpServletRequest.setAttribute(String, Object)}; calling this method ensures
access to the named models from EL expressions.

A view returned by a controller method represents a relative path within an
application archive. Implementations MUST resolve view paths relative to 
\code{ViewEngine.DEFAULT\_VIEW\_FOLDER}, which is set to \code{/WEB-INF/views/}.
Locating views under \code{WEB-INF} ensures that they are not  
directly accessible to clients as static resources.

