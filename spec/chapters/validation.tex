\chapter{Exception Handling}
\label{exception_handling}

This chapter discusses exception handling in the \mvc\ API. Exception handling is based
on the underlying mechanism provided by \jaxrs, but with additional support for 
binding and validation exceptions that are common in \mvc\ frameworks.

\section{Exception Mappers}
\label{exception_mappers}

The general exception handling mechanism in \mvc\  controllers is identical to that defined
for resource methods in the \jaxrs\ specification. In a nutshell, applications can 
implement exception mapping providers for the purpose of converting exceptions to 
responses. If an exception mapper is not found for a particular exception type, 
default rules apply that describe how to process the exception depending on whether
it is checked or unchecked, and using additional rules for
the special case of a \WebAppExc\ that includes a response. The reader is referred
to the \jaxrs\ specification for more information.

Let us consider the case of a \ConstVioExc\ that is thrown as a result of a bean validation
failure:

\begin{listing}{1}
@Controller
@Path("form")
public class FormController {

    @POST
    public Response formPost(@Valid @BeanParam FormDataBean form) {
        return Response.status(OK).entity("data.jsp").build();    
    }
}
\end{listing}

The method \code{formPost} injects a bean parameter of type \code{FormDataBean}
which, for the sake of the example, we assume includes validation constraints
such as \code{@Min(18)}, \code{@Size(min=1)}, etc. The presence of \code{@Valid} triggers
validation of the bean on every HTML form post; if validation fails, a 
\ConstVioExc\ (a subclass of \ValExc) is thrown. 

An application can handle the exception by including an exception mapper as follows:

\begin{listing}{1}
public class FormViolationMapper 
             implements ExceptionMapper<ConstraintViolationException> {

    @Inject
    private ErrorDataBean error;

    @Override
    public Response toResponse(ConstraintViolationException e) {
        final Set<ConstraintViolation<?>> set = e.getConstraintViolations();
        if (!set.isEmpty()) {
            // fill out ErrorDataBean ...
        }
        return Response.status(Response.Status.BAD_REQUEST)
                       .entity("error.jsp").build();
    }
}
\end{listing}

This exception mapper updates an instance of \code{ErrorDataBean} and returns 
the \code{error.jsp} view 
(wrapped in a response as required by the method signature) with the intent to provide a 
human-friendly description of the exception.

Even though using exception mappers is a convenient way to handle exceptions in general, 
there are cases in which finer control is necessary. The mapper defined above will be
invoked for all instances of \ConstVioExc\ thrown in an application. Given that 
applications may include several form-post controllers, handling all exceptions using
a single method makes it difficult to provide controller-specific customizations.
Moreover, exception mappers do not get access to the (partially valid) bound
data, or \code{FormDataBean} in the example above.

\section{Validation Exceptions}
\label{validation_exceptions}

MVC provides an alternative exception handling mechanism that is specific for the 
use case described in Section~\ref{exception_mappers}. Rather than funnelling 
exception handling into a single location while providing no access to the bound
data, controller methods may opt to act as exception handlers as well. In other words, 
controller methods can get called even if parameter validation fails as long as
they declare themselves capable of handling errors.

A controller class that, either directly or indirectly via inheritance, defines 
a field or a property of type 
\code{javax.mvc.binding.BindingResult} will have its controller methods
called even if a validation error is encountered while validating parameters. 
Implementations MUST introspect the controller bean for a field or a property of
this type to determine the correct semantics; fields and property setters
of this type MUST be annotated with \code{@Inject} to guarantee proper bean
initialization \assertref{validation-result}.

Let us revisit the example from Section~\ref{exception_mappers},
this time using the controller method as an exception handler:

\begin{listing}{1}
@Controller
@Path("form")
public class FormController {

    @Inject
    private BindingResult br;
    
    @Inject
    private ErrorDataBean error;

    @POST
    @ValidateOnExecution(type = ExecutableType.NONE)
    public Response formPost(@Valid @BeanParam FormDataBean form) {
        if (br.isFailed()) {
            // fill out ErrorDataBean ...
            return Response.status(BAD_REQUEST).entity("error.jsp").build();
        }
        return Response.status(OK).entity("data.jsp").build();    
    }
}
\end{listing}

The presence of the injection target for the field 
\code{br} indicates to an implementation that controller methods in this
class can handle errors. As a result, methods in this class
that validate parameters should call \code{br.isFailed()} to verify if
validation errors were found.~\footnote{The \code{ValidateOnExecution} 
annotation is necessary to ensure that CDI and BV do not abort the
invocation upon detecting a violation. Thus, to ensure the correct semantics, 
validation must be performed by the JAX-RS implementation before the method 
is called.}

The class \code{BindingResult} provides methods to get detailed information 
about any violations 
found during validation. Instances of this class are always in request
scope; the reader is referred to the Javadoc for more information.

As previously stated, properties of type  \code{BindingResult}
are also supported. Here is a modified version of the example in which
a property is used instead:

\begin{listing}{1}
@Controller
@Path("form")
public class FormController {

    private BindingResult br;
    
    public BindingResult getBr() {
        return vr;
    }

    @Inject
    public void setBr(BindingResult br) {
        this.br = br;
    }
    ...
}
\end{listing}

Note that the \code{@Inject} annotation has been moved from the field to the setter,
thus ensuring the bean is properly initialized by CDI when it is
created. Implementations MUST give precedence to a property (calling its getter and
setter) over a field if both are present in the same class.
 
\section{Binding Exceptions}
\label{binding_exceptions}

As suggested by its name, instances of \code{BindingResult} also track any 
binding errors that occur while mapping request parameters to Java
types. Binding errors are discovered even before validation takes place. An
example of a binding error is that of a query parameter bound to an {\tt int} whose
value cannot be converted to that type. 

 \jaxrs\ uses the notion of a parameter converter
to provide extension points for these conversions; if none are specified for
the type at hand, a set of default parameter converters is available. Regardless
of where the parameter converter is coming from, a failure to carry out a 
conversion results in an \code{IllegalArgument\-Exception} thrown and, typically,
a 500 error code returned the client. As explained before, applications can provide
an exception mapper for \code{IllegalArgumentException} but this may be
insufficient when error recovery using controller-specific logic is required. 

Controllers can call the same \code{isFailed} method to check for binding errors
---the method returns true if at least one error of either kind is found.
Additional methods in the \code{BindingResult} type allow to get specific 
information related to binding errors. See the Javadoc for more information.
