\chapter{Exception Handling}
\label{exception_handling}

This chapter discusses exception handling in the \mvc\ API. Exception handling is based
on the underlying mechanism provided by \jaxrs, but with additional support for 
validation exceptions that are common in HTML form posts.

\section{Exception Mappers}
\label{exception_mappers}

The general exception handling mechanism in MVC controllers is identical to that defined
for resource methods in the \jaxrs\ specification. In a nutshell, applications can 
implement exception mapping providers for the purpose of mapping exceptions to 
responses. If an exception mapper is not found for a particular exception type, 
default rules apply that describe how to process the exception depending on whether
it is a checked or an unchecked exception, and using additional rules for
the special case of a \WebAppExc\ that includes a response. The reader is referred
to the \jaxrs\ specification for more information.

Let us consider the case of a \ValExc\ that is thrown as a result of a bean validation
failure:

\begin{listing}{1}
@Controller
@Path("form")
@Produces("text/html")
public class FormController {

    @POST
    public Response formPost(@Valid @BeanParam FormDataBean form) {
        return Response.status(OK).entity("data.jsp").build();    
    }
}
\end{listing}

The method \code{formPost} injects a bean parameter of type \code{FormDataBean}
which, for the sake of the example, we assume includes validation constraints
such as \code{@Min(18)}, \code{@Size(min=1)}, etc. The presence of \code{@Valid} triggers
validation of the bean on every HTML form post; if validation fails, a 
\ConstVioExc\ (a subclass of \ValExc) is thrown. 

An application can handle the exception by including an exception mapper as follows:

\begin{listing}{1}
class FormViolationMapper implements 
                          ExceptionMapper<ConstraintViolationException> {

    @Inject
    private ErrorDataBean error;

    @Override
    public Response toResponse(ConstraintViolationException e) {
        final Set<ConstraintViolation<?>> set = e.getConstraintViolations();
        if (!set.isEmpty()) {
            final ConstraintViolation<?> cv = set.iterator().next();
            // fill out ErrorDataBean ...

        }
        return Response.status(Response.Status.BAD_REQUEST)
                       .entity("error.jsp").build();
    }
}
\end{listing}

This exception mapper fills out an \code{ErrorDataBean} and returns the \code{error.jsp} view, 
wrapped in a response as required by the method signature, with the intent to provide a human-friendly 
description of the exception.

Even though using exception mappers is a convenient way to handle exceptions in general, 
there are cases in which finer control is necessary. The mapper defined above will be
invoked for all instances of \ConstVioExc\ thrown in an application. Given that 
applications may include several form-post controllers, handling all exceptions using
a single method makes it difficult to provide controller-specific customizations.
Moreover, exception mappers do not get access to the (likely partially valid) bound
data, or \code{FormDataBean} in the example above.

\section{Validation}
\label{validation}

MVC provides an alternative exception handling mechanism that is specific for the 
use case described in Section~\ref{exception_handling}. Rather than funnelling 
exception handling into a single location while providing no access to the bound
data, controller methods may opt to act as exception handlers as well. In other words, 
controller methods can get called even if parameter validation fails as long as
they declare themselves capable of handling errors.

A controller class that defines a field or a property of type 
\code{javax.mvc.validation.ValidationResult} will have its controller methods
called even if a validation error is encountered while validating parameters. 
Let us revisit the example from Section~\ref{exception_handling},
this time using the controller method as an exception handler.

\begin{listing}{1}
@Controller
@Path("form")
@Produces("text/html")
public class FormController {

    @Inject
    private ValidationResult vr;
    
    @Inject
    private ErrorDataBean error;

    @POST
    @ValidateOnExecution(type = ExecutableType.NONE)
    public Response formPost(@Valid @BeanParam FormDataBean form) {
        if (vr.isFailed()) {
            final Set<ConstraintViolation<?>> set = vr.getAllViolations();
            // fill out ErrorDataBean ...
            return Response.status(BAD_REQUEST).entity("error.jsp").build();
        }
        return Response.status(OK).entity("data.jsp").build();    
    }
}
\end{listing}

The presence of the injection target for the field 
\code{vr} indicates to an implementation that controller methods in this
class can handle validation errors. As a result, methods in this class
that validate parameters should call \code{vr.isFailed()} to verify if
validation errors were found.~\footnote{The \code{ValidateOnExecution} 
annotation is necessary to ensure that CDI calls the method even after 
a validation failure is encountered. Thus, to ensure the correct semantics, 
validation  must be done by the JAX-RS implementation before the method 
is called.}

The class \code{javax.mvc.validation.ValidationResult}
provides methods to get detailed access about any violations 
found during validation. Instances of this class are always in request
scope; the reader is referred to the javadoc for more information.

As previously stated, properties of type \code{javax.mvc.validation.ValidationResult}
are also supported. Here is a modified version of the example in which
a property is used instead:

\begin{listing}{1}
@Controller
@Path("form")
@Produces("text/html")
public class FormController {

    private ValidationResult vr;
    
    @Inject
    private ErrorDataBean error;
    
    public ValidationResult getVr() {
        return vr;
    }

    @Inject
    public void setVr(ValidationResult vr) {
        this.vr = vr;
    }

    @POST
    @ValidateOnExecution(type = ExecutableType.NONE)
    public Response formPost(@Valid @BeanParam FormDataBean form) {
        if (vr.isFailed()) {
            final Set<ConstraintViolation<?>> set = vr.getAllViolations();
            // fill out ErrorDataBean ...
            return Response.status(BAD_REQUEST).entity("error.jsp").build();
        }
        return Response.status(OK).entity("data.jsp").build();    
    }
}
\end{listing}

Note that the \code{@Inject} has been moved from the field to the setter,
thus ensuring the bean is properly initialized by CDI when it is
created. Implementations MUST give precedence to a property over a field
if both are present in the same class.
 

