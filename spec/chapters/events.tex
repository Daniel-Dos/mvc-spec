\chapter{Events}
\label{events}

This chapter introduces a mechanism by which MVC applications can be informed 
of important events that occur while processing a request. This mechanism 
is based on CDI events that can be fired by implementations and observed by 
applications.

\section{Observers}
\label{observers}

The package \code{javax.mvc.event} defines a number of event types that MUST be 
fired by implementations during the processing of a request \assertref{event-firing}. 
Implementations MAY
extend this set and also provide additional information on any of the events defined
by this specification. The reader is referred to the implementation's documentation
for more information on event support.

Observing events can be useful for applications to learn about the lifecycle of a 
request, perform logging, monitor performance, etc. The events
\code{BeforeControllerEvent} and \code{AfterControllerEvent} are fired around
the invocation of a controller; applications can monitor these events using an 
observer as shown next.

\begin{listing}{1}
@ApplicationScoped
public class EventObserver {

    public void onBeforeController(@Observes BeforeControllerEvent e) {
        System.out.println("URI: " + e.getUriInfo().getRequestURI());
    }
    
    public void onAfterController(@Observes AfterControllerEvent e) {
        System.out.println("Controller: " + 
            e.getResourceInfo().getResourceMethod());
    }
}
\end{listing}

Observer methods in CDI are defined using the \code{@Observes} annotation on 
a parameter position.
The class \code{EventObserver} is a CDI bean in application scope whose methods
\code{onBeforeController} and \code{onAfterController} are called 
before and after a controller is called. 

Every event generated must include a
unique ID whose getter is defined by \code{MvcEvent}, the base type for all
events. Moreover, each event includes additional information that is specific to
the event; for example, the events shown in the example above allow applications
to get information about the request URI and the resource (controller) selected.

Chapter \ref{view_engines} describes the algorithm used by implementations to select a
specific view engine for processing; after a view engine is selected, the method
\code{processView} is called. The events \code{BeforeProcessViewEvent}
and \code{AfterProcessViewEvent} are fired around this call and can be observed
in a similar manner:

\begin{listing}{1}
@ApplicationScoped
public class EventObserver {

    public void onBeforeProcessView(@Observes BeforeProcessViewEvent e) {
        ...
    }
    
    public void onAfterProcessView(@Observes AfterProcessViewEvent e) {
        ...
    }
}
\end{listing}

To complete the example, let us assume that the information about the selected
view engine needs to be conveyed to the client. To ensure that this information
is available to a view returned to the client, the \code{EventObserver} class
can inject and update the same request-scope bean accessed by the view:

\begin{listing}{1}
@ApplicationScoped
public class EventObserver {

    @Inject
    private EventBean eventBean;

    public void onBeforeProcessView(@Observes BeforeProcessViewEvent e) {
        eventBean.setView(e.getView());
        eventBean.setEngine(e.getEngine());
    }
    
    ...
}
\end{listing}

For more information about the interaction between views and models, the reader
is referred to Section~\ref{models}. 

CDI events fired by implementations are {\em synchronous},
so it is recommended that applications carry out only simple tasks in 
their observer methods, avoiding long-running computations as well as blocking calls.
For a complete list of events, the reader is referred to the Javadoc for the 
\code{javax.mvc.event} package.

Event reporting requires the MVC implementations to create event objects before
firing. In high-throughput systems without any observers the number of unnecessary
objects created may not be insignificant. For this reason, it is RECOMMENDED
for implementations to consider smart firing strategies when no observers
are present.


