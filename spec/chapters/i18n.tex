\chapter{Internationalization}
\label{i18n}

This chapter introduces the notion of a {\em request locale} and describes how \mvc\
handles internationalization and localization.

\section{Introduction}
\label{i18n_introduction}

Internationalization and localization are very important concepts for any web application 
framework. Therefore \mvc\ has been designed to make supporting multiple languages and
regional differences in applications very easy.

\mvc\ defines the term {\em request locale} as the locale which is used for any 
locale-dependent operation within the lifecycle of a request. The request locale 
MUST be resolved exactly once for each request using the resolving algorithm described in
Section \ref{i18n_resolving_algorithm}.

These locale-dependent operations include, but are not limited to:

\begin{enumerate}
\item Data type conversion as part of the data binding mechanism.
\item Formatting of data when rendering it to the view.
\item Generating binding and validation error messages in the specific language.
\end{enumerate}

The request locale is available from \code{MvcContext} and can be used by controllers, 
view engines and other components to perform operations which depend on the current locale 
\assertref{request-locale-context}. The example below shows a controller that uses the 
request locale to create a \code{NumberFormat} instance.

\begin{listing}{1}
@Controller
@Path("/foobar")
public class MyController {

    @Inject
    private MvcContext mvc;
    
    @GET
    public String get() {
    
        Locale locale = mvc.getLocale();
        NumberFormat format = NumberFormat.getInstance(locale);
    
    }

}
\end{listing}

The following sections will explain the locale resolving algorithm and the default
resolver provided by the \mvc\ implementation.

\section{Resolving Algorithm}
\label{i18n_resolving_algorithm}

The {\em locale resolver} is responsible to detect the request locale for each request
processed by the \mvc\ runtime. A locale resolver MUST implement the
\code{javax.mvc.locale.LocaleResolver} interface which is defined like this:

\begin{listing}{1}
public interface LocaleResolver {

    Locale resolveLocale(LocaleResolverContext context);

}
\end{listing}

There may be more than one locale resolver for a \mvc\ application. Locale resolvers
are discovered using CDI \assertref{extension-resolvers}. Every CDI bean
implementing the \code{LocaleResolver} interface and visible to the application 
participates in the locale resolving algorithm.

Implementations MUST use the following algorithm to resolve the request locale for
each request \assertref{resolve-algorithm}:

\begin{enumerate}
\item Obtain a list of all CDI beans implementing the \code{LocaleResolver} interface
 visible to the application's \code{BeanManager}.
\item Sort the list of locale resolvers in descending order of priority using the
 integer value from the \Priority\ annotation decorating the resolver class. If no
 \Priority\ annotation is present, assume a default priority of \code{1000}.
\item Call \code{resolveLocale()} on the first resolver in the list. If the resolver
 returns \code{null}, continue with the next resolver in the list. If a resolver 
 returns a non-null result, stop the algorithm and use the returned locale as the request 
 locale. 
\end{enumerate}

Applications can either rely on the default locale resolver which is described
in Section \ref{i18n_default_resolver} or provide a custom resolver which implements
some other strategy for resolving the request locale. A custom strategy could for
example track the locale using the session, a query parameter or the server's
hostname.

\section{Default Locale Resolver}
\label{i18n_default_resolver}

Every \mvc\ implementation MUST provide a default locale resolver with a priority of \code{0}
which resolves the request locale according to the following algorithm 
\assertref{default-locale-resolver}:

\begin{enumerate}
\item First check whether the client provided an \code{Accept-Language} request header.
 If this is the case, the locale with the highest quality factor is returned as
 the result. 
\item If the previous step was not successful, return the system default locale of the
 server.
\end{enumerate}

Please note that applications can customize the locale resolving process by providing
a custom locale resolver with a priority higher than \code{0}.
See Section \ref{i18n_resolving_algorithm} for details.
